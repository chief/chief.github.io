% !TEX encoding = UTF-8
\documentclass[12pt]{article}
\usepackage[LGR]{fontenc}
\usepackage[greek,english]{babel}
\usepackage[utf8x]{inputenc}
\title{Στo ρίνγκ}
\date{Αθήνα 2018}
\author{Γιώργος \latintext Timetrap}

\usepackage{verse}

\begin{document}
\maketitle
\newpage

\poemtitle{Αδυναμίες}
\settowidth{\versewidth}{Τόσο πια που ντρέπεται στον εαυτό του να μιλάει}
\begin{verse}[\versewidth]
  Άλλαξε η αδυναμία του, στο δυνατότερο που έχει \\
  οι φυλακές του ελεύθερες, υψίστης ασφαλείας \\
  και οι σιωπές του γράφτηκαν, στα αποφθέγματά του.

  Άλλαξε και αυτός, ο δυνατός ο ελεύθερος, \\
  ο γελαστός. \\
  Άλλαξε και αυτός, ο θλιμμένος ο άπραγος, \\
  ο ισχνός.

  Τόσο πια που ντρέπεται στον εαυτό του να μιλάει, \\
  τόσο πια που ντρέπεται στον εαυτό του να μιλάει, \\
  τόσο πια που ντρέπεται στον εαυτό του να μιλάει!

  Και έτσι ξαφνικά αρχίζει να χιονίζει \\
  και έτσι ξαφνικά αρχίζει ο χιονιάς. \\
  Κανείς δεν πάει στην δουλειά \\
  και αυτός χαίρεται κρυφά.

  Και έτσι τυχαία πετυχαίνει έναν άγνωστο \\
  και έτσι μοιραία συναντάει έναν ξένο. \\
  Γίναν φίλοι σε μια νύχτα \\
  και αυτός χαίρεται στ΄ αλήθεια.

  Και έτσι ερωτικά πλαγιάζει με ερωμένη του \\
  και έτσι ερωτικά λιώνει στο κορμί της. \\
  Και ήταν σαν και τότε \\
  που άλλαξε το παιδί.

  Μα το χιόνι έλιωσε, ο φίλος βαρετός \\
  και η ερωμένη έγινε θρησκευτικός δεσμός.

  Αναρωτιέται τι φταίει \\
  βασανίζεται λύση να βρει. \\
  Γονατίζει και κλαίει

  Τότε ήταν που η αδυναμία του έγινε δύναμη \\
  φόβου αδρεναλίνη ροή εγρήγορσης \\
  και η απάθεια με μιας έγινε ενέργεια. \\

  Τότε ήταν που επιχείρησε πρώτη \\
  και τέλευταία του \\
  έξοδο.

  Τότε ήταν που ...

  φορτώνει βάρκα ξέκινά και αποφασίζει απλά \\
  ναυαγός να βρεθεί στο βράχο του ``θέλω'' \\
  ναυαγός να βρεθεί στο βράχο του ``θέλω'' \\
  στην νήσο του πρέπει δεν ξαναγυρνά \\
  στην νήσο του πρέπει όχι ξανά!
\end{verse}

\newpage

\poemtitle{Αγία Τέχνη}
\settowidth{\versewidth}{δελτα του λάμδα}
\begin{verse}[\versewidth]
  Γαμα στο ζητα \\
  και πες σε μι \\
  δελτα του λαμδα \\
  και ανατροπι
\end{verse}

\newpage

\poemtitle{Η Γενιά μας}
\settowidth{\versewidth}{συγκρούστηκε με κράσπεδα παράλογης}
\begin{verse}[\versewidth]
  Η γενιά μας, \\
  συγκρούστηκε με κράσπεδα παράλογης ταχύτητας \\
  βυθίστηκε σε κριθάριους ποταμούς \\
  γεύτηκε τις στρυχνίνες \\
  κέντησε πλεκτά με θανατερά βελόνια \\
  έφαγε τους καρπούς της άρνησης \\
  τσάκιστηκε σε βράχους νικοτίνης \\
  μαγεύτηκε από το φως της τηλεόρασης \\
  τυφλώθηκε από τα ίδια της τα χέρια \\
  ξεχάστηκε στα βιντεοπαιχνίδια \\
  ξεβράστηκε στην στεριά \\
  γελάστηκε από γέροντες \\
  δαγκώθηκε από σκυλιά \\
  χτυπήθηκε από προστάτες \\
  εκλιπάρησε παρακάλεσε φίλησε \\
  για μια θέση μια δουλειά μια ποδιά \\
  έχασε από πλούσιους \\
  προδόθηκε από έρωτες \\
  περιφρονήθηκε \\
  κουρελιάστηκε \\
  διασύρθηκε.

  Η γενιά μας, \\
  ήταν η μόνη που άλεσε ψωμί από τους ανεμόμυλούς της
\end{verse}

\newpage

\poemtitle{Η χαρά}
\settowidth{\versewidth}{ίπταται αναπαίσθητα σχεδόν}
\begin{verse}[\versewidth]
  Η χαρά της ζωής δεν εγκλωβίζεται \\
  σε χαρακώματα προσωπικών επαφών \\
  δεν συλλαμβάνεται σε απόχες \\
  πανούργων ανθρωποδίφων \\
  που ως γνήσιοι συλλέκτες \\
  επιθυμούν σε σφαγιστό δοχείο \\
  να την δουν

  Η χάρα μου για την ζωή \\
  ίπταται αναπαίσθητα σχεδόν αόρατα \\
  πάνω από τις μάχες σας \\
  και ξεφεύγει των ειδικών \\
  φοβούμενη να χαρακτηριστεί \\
  είδος προς εξαφάνιση, τι λιπι!
\end{verse}

\newpage

\poemtitle{Μεθύσι}
\settowidth{\versewidth}{ξερνάμε λέξεις σαν από μεθύσι με φτηνό}
\begin{verse}[\versewidth]
  ξερνάμε λέξεις σαν από μεθύσι με φτηνό κρασί \\
  που η γεύση πικρή η ζάλη ο σκοπός \\
  και η απόλαυση ο παρδαλός χορός

  ζαλισμένοι στις σκέψεις μας \\
  σε μια κοινωνία φτηνά πανηγυρική \\
  με δωρεάν ποτά να σερβίρονται ατέλειωτα

  το μόνο που ζητούμε είναι παυσίπονα \\
  για να συνεχίσουμε
\end{verse}

\newpage

\poemtitle{Ο φίλος}
\settowidth{\versewidth}{τους δασκάλους μισούσε}
\begin{verse}[\versewidth]
  είχα έναν φίλο που \\
  τον λέγαν αλήτη \\
  και στο σπίτι \\
  δεν μπορούσε να μείνει

  τους δασκάλους μισούσε \\
  και τσίχλες μασούσε \\
  σχεδόν χλευαστικά

  στα παιδιά δεν μιλούσε \\
  και τις ώρες περνούσε \\
  κάπου μοναχικά
\end{verse}

\newpage

\poemtitle{Πάνω από όλα}
\settowidth{\versewidth}{κατω απ' όλα ο άνθρω}
\begin{verse}[\versewidth]
  πάνω απ'όλα ασφάλεια \\
  σκούζει ο φύλακας \\
  στον φυλακισμένο

  πάνω απ' όλα τάξη λέει \\
  ο αστυνόμος \\
  στην γιορτή μας

  πάνω απ΄όλα κέρδος \\
  γνέφει ο γιάπης \\
  στον ζητιάνο

  κάτω απ' όλα ο άνθρωπος \\
  τραγουδούν όλοι μαζί τους

  κάτω απ' όλα ο άνθρωπος \\
  τραγουδούν όλοι μαζί
\end{verse}

\newpage

\poemtitle{Πίσω από εμάς}
\settowidth{\versewidth}{πίσω από το δίχως όνομα}
\begin{verse}[\versewidth]
  πίσω από το πρόσωπο \\
  πίσω από την ένοπλη φωνή μας \\
  πίσω από το δίχως όνομα όνομά μας \\
  πίσω από εμάς που εσείς βλέπετε \\
  είμαστε εσείς
\end{verse}

\newpage

\poemtitle{Τέλος}
\settowidth{\versewidth}{σαν δάκρυ}
\begin{verse}[\versewidth]
  Χιονίζει \\
  και κρυώνω \\
  αναμεσά μας \\
  Έχω φρακάρει \\
  το παγερό σου \\
  βλέμμα με \\
  σαστίζει \\
  Σιγά σιγά \\
  ψύχος διαπερνά \\
  από τα μάτια \\
  και η επαφή σου \\
  μεταδίδει, \\
  έτσι που όμοιοι \\
  σαν δύο νιφάδες \\
  ξαπλώνοντας στο \\
  δρόμο μουδιασμένοι \\
  θα βλέπουμε ο ένας \\
  την μορφή του άλλου \\
  να λιώνει και \\
  σαν δάκρυ θα αποχωριστούμε.
\end{verse}

\newpage

\poemtitle{Τύψη}
\settowidth{\versewidth}{έκανα φόνο}
\begin{verse}[\versewidth]
  Χτες το βράδυ \\
  έκανα φόνο \\
  έπνιξα τύψη \\
  και μετανιώνω.
\end{verse}

\newpage

\poemtitle{Το άροτρο}
\settowidth{\versewidth}{Ποιό είναι το δικό μου άροτρο}
\begin{verse}[\versewidth]
  Ποιό είναι το δικό μου άροτρο; \\
  Ποιός ο δικός μου δρόμος; \\
  Ποιό μονοπάτι έσκαψα και κάλυψα τα ίχνη; \\
  Στους περρισότερους, \\
  βλέπω μές την θωριά τους \\
  το βλέμμα εκείνου του \\
  ιερού καθώς το λένε ζώο \\
  μιας κ' αγωνίζονται να \\
  κρατηθούν σε ευθυγραμμισμένες \\
  παράλληλες πορείες ζωής \\
  με φόβο χαραγμένες. \\
  Μα αλοίμονο αν στην ζωή \\
  το ζύγι σε πλακώσει! \\
  Βάσανο γίνεται μετά και \\
  πλάκωμα και κόπος \\
  τα βλέφαρα μονάχα τους \\
  θα κλείνουνε συνέχεια \\
  μην αντικρύσει η ματιά \\
  την άθλια μιζέρια. \\
  Μα ακόμη πιο οδυνηρό είναι \\
  αν υποθέσεις πως είναι \\
  ανάλαφρο το άροτρο που έχεις! \\
  Μην γελαστείς από  αυτούς \\
  που θέν να σου προσφέρουν \\
  παυσίπονα κ' ηρεμιστικά \\
  ντυμένα με εικασίες \\
  Το κάνουν γιατί έχασαν \\
  πια τον δικό τους δρόμο \\
  και αν τους πιστέψεις \\
  πιθανό, από την ελαφρότητά σου \\
  να μην σκαλίσεις την ζωή βαθιά \\
  που είναι όλη δικιά σου.
\end{verse}

\newpage

\poemtitle{To ασανσέρ}
\settowidth{\versewidth}{έχω κλειστεί στο ασανσέρ και}
\begin{verse}[\versewidth]
  έχω κλειστεί στο ασανσέρ και περιμένω \\
  εγκλωβιστεί σε χίλια στεγανά \\
  επιλογές που μου προσφέρονται πατάω \\
  μόνον με αφήνουνε, σχεδόν αμήχανά

  κενό μετέωρο στον θάλαμο κλεισμένο \\
  κενό μετέωρο σε κάψουλα στενή \\
  κενό μετέωρο στον θάλαμο, δεν βγαίνω \\
  αντέχω τα βασανιστήρια

  ζωή

  καμια εξέλιξη μα ούτε καν και πτώσις, \\
  έστω μια πρόσκρουση με λάμψη φωτεινή \\
  κανένα όνειρο μα ούτε και Eφιάλτης \\
  μα που θα πάει θα "πιάσει" το κουμπί

  μετέωροι είμαστε σε ασανσερ σου λέω \\
  και από πάνω το τσιγκέλι της ζωής \\
  αγκομαχώντας οι μέρες μία μία \\
  κουτσανασαίνουν στην κάθε τους

  στιγμή
\end{verse}

\newpage

\poemtitle{Τρέξε}
\settowidth{\versewidth}{και η παιδικη σου αφελεια χαθει}
\begin{verse}[\versewidth]
  Οταν η σιωπη, σου γινει συνηθεια \\
  οταν η καρδια σου κλειδώσει \\
  και η παιδικη σου αφελεια χαθει \\
  τοτε θα αρχισεις να τρεχεις \\

  Οταν πατησεις ολους τους ορκους \\
  που με πεισμα σφιχτα υποσχεθηκες \\
  και στο θεατρο μπεις της υποκρισιας \\
  τοτε θα σε χειροκροτησουν

  Ολοι αυτοι που τρεχουν μαζι σου \\
  Ολοι αυτοι που τρεχουν μαζι σου \\
  Ολοι αυτοι που μοιαζουν μαζι σου \\
  αλλα δεν θα ειναι ποτε μαζι σου

  Οταν οι γελωτοποιοι σε μεθυσουν \\
  και οι παλιατσοι σε μολυνουν με ντοπες \\
  εγωισμου αυταπατης και ψεμματος \\
  τοτε θα επιδιωκεις να τερματισεις πρωτος

  Να πηδηξεις ολες τις γκομενες \\
  Να κοροιδεψεις ολους τους φιλους σου \\
  Να ειρωνευτεις να λοιδορισεις \\
  Να μαχαιρωνεις πισωπλατα

  Πιο δυνατα ! \\
  Πιο καλα ! \\
  Για λεφτα ! \\
  Για ανοησια για μαγκια !

  Γιατι δεν σταθηκες ικανος να βγεις \\
  απο τον δρομο που σου χαραξαν \\
  Μικρο ανθρωπακι !

  Γιατι τερματιζεις πρωτος τωρα \\
  και εισαι μονος \\
  Γιατι δεν υπαρχουν φωτα και καμερες \\
  Και ολοι αυτοι που σε χειροκροτουσαν \\
  τωρα σε μισουν

  Πως την θεση σου γυρευουν και \\
  σε κατηγορουν. \\
  Μικρο ανθρωπακι !

  Γιατι ξεχασες τα λογια σου, \\
  Γιατί ξέχασες τα λόγια σου, \\
  Το νου σου ! \\
  Γιατι τον εχασες

  Ευχαριστω δεν θα τρεξω μαζι σου.
\end{verse}

\newpage

\poemtitle{Στρουθοκάμηλος}
\settowidth{\versewidth}{έχω παράλυση δεν βρίσκω λύση σε τίποτα}
\begin{verse}[\versewidth]
  Εχω μουδιάσει κάτω από την άμμο και περιμένω \\
  να μην δω φόβο κάτω από την άμμο σαν χαζοπούλι

  έχω παράλυση δεν βρίσκω λύση σε τίποτα \\
  ολα βουνο μοιάζουν ακόμα και τ' ασήμαντα

  και κάθε πρωι χτυπώ το κεφάλι μου στον τοίχο \\
  όλες οι εξεγερμένες ημικρανίες μου ενάντια

  στο άπραγο εγώ διψουνε για δρασεις \\
  κάθε πρωι χτυπώ το κεφάλι στον τοίχο

  και τις πνίγω
\end{verse}

\newpage

\poemtitle{Η πέτρα}
\settowidth{\versewidth}{τρύπες ανοίγω στα κεφάλια ανθρώπων σαν την}
\begin{verse}[\versewidth]
  έχω μια πέτρα σφιχτά την κρατάω \\
  την λέω ρουτίνα και από συνήθεια
  την κουβάλαω

  τρύπες ανοίγω στα κεφάλια ανθρώπων σαν την πετάω \\
  τρύπες και η φαντασία \\
  τους στερεύει \\

  μια πέτρα της λήθης στα χέρια μου έχω και διαγράφω \\
  φίλους, συντρόφους, λύπες, φωτιές, μελωδίες και γλέντια \\
  ρίχνω μια πέτρα σε μένα, και χάνομαι

  μα μενω μόνος \\
  με τις συνηθειες \\
\end{verse}

\newpage

\poemtitle{Μίλα μου}
\settowidth{\versewidth}{χρόνια βουβός αντικρίζω}
\begin{verse}[\versewidth]
  είμαι χωρίς μιλιά \\
  χρόνια βουβός αντικρίζω \\
  τις συσκευές να φορτίζονται \textperiodcentered \\
  μίλα μου
\end{verse}

\newpage

\poemtitle{Μεθύσι 2}
\settowidth{\versewidth}{γεμάτα πρακτορεία με νέους ανάπηρους}
\begin{verse}[\versewidth]
  δουλειά σπίτι καφενείο θάνατος \\
  ζωή σε ξέφρενη πορεία παρακμής \\
  γεμάτα πρακτορεία με νέους ανάπηρους \\
  ελπίδα σάπια σε κονσέρβα ηδονής

  εγκλωβισμένοι σε κοινωνία ωμά θεαματική \\
  με ατέλειωτα σκουπίδια να σερβίρονται συνέχεια \\
  το μόνο που ζητάμε είναι παυσίπονα \\
  να πέσουμε νεκροί ως το επόμενο πρωϊ
\end{verse}

\newpage

\poemtitle{Βαλίτσες}
\settowidth{\versewidth}{σχέδια, όνειρα, έρωτες και χαρές}
\begin{verse}[\versewidth]
  φτιάχνω βαλίτσες και βάζω σε τάξη \\
  σχέδια, όνειρα, έρωτες και χαρές \\
  ανοίγω ντουλάπες τις σπρώχνω στην άκρη \\
  πέφτω για ύπνο νωρίς γιατί έχω δουλειές

  γύρω μου όλα φωτιά \\
  ξυπνώ και ζω σε παραλήρημα \\
  απορώ πού φταίω εγώ \\
  ομολογώ δεν έκανα τίποτα \\
  είμαι αθώος \\
  ομολογώ δεν κάνω πια τίποτα \\
  εκείνος φταίει μάλλον αυτός

  γυρνάω στο σπίτι ανοίγω οθόνες \\
  χαζεύω την βία και αυτή μου γελά \\
  στην άκρη της σκέψης τα όνειρα μου \\
  αχνά τα θυμάμαι και αποκοιμιέμαι γλυκά
\end{verse}

\newpage

\poemtitle{Πόσο πάει μια ψυχή}
\settowidth{\versewidth}{και αυτός ο Κινέζος κλεισμένος}
\begin{verse}[\versewidth]
  πόσο πάει μια ψυχή \\
  ενός παιδιού από Αφρική \\
  που σε πλοίο μπαρκάρει \\
  για την Κάτω Ιταλιά

  πόσο να 'ναι η τιμή \\
  ενός σκλάβου νέου Πέρση \\
  για ατέλειωτη δουλειά σε \\
  χωράφια κόκκινα

  και αυτός ο Κινέζος κλεισμένος \\
  στον ουρανό εκεί πάνω \\
  σε ένα κουτί πόσα μερόνυχτα σερί \\
  να δουλεύει άραγες

  και τι γίνεται αν ξέρουν όλοι αυτοί \\
  όντως μπάλα να κλωτσούν;
\end{verse}

\newpage

\poemtitle{Αύριο}
\settowidth{\versewidth}{Περιμέναμε ένα αύριο αλλιώτικο}
\begin{verse}[\versewidth]
  Σου λέω κουράστηκα \\
  Δεν ξέρω πως, πως να συνεχίσω \\
  Εξαϋλωθηκα, σε χρόνο νεκρό \\
  σε τόπο κενό

  σε χρόνο νεκρό εξοντώθηκα \\
  σε τόπο κενό πλανιέμαι \\
  σε χρόνο νεκρό εξοντώθηκα \\
  σε τόπο κενό πλανιέμαι

  Και εσύ μου λες αύριο \\
  Θα ήταν καλύτερα \\
  Αν μου πιανες το χέρι \\
  Και ήμασταν μαζί \\
  Στα δύσκολα που είχαν έρθει \\
  από καιρό ήταν εδώ

  Περιμέναμε ένα αύριο αλλιώτικο \\
  Μα ήταν πέραν του αναμενόμενου \\
  Σου τα πα και πριν τεμαχίστηκα \\
  σε χίλιες στιγμές \\
  σε χίλιες ζωές \\
  συντρίμια

  Έγινα σκιά \\
  δεν μπορείς \\
  Σου δίνω το χέρι να το πιάσεις \\
  Μόν ίσκιο από του \\
  αύριο την όψη.

  (13/10/2013)
  για μια άσχημη μέρα
\end{verse}

\newpage

\latintext

\poemtitle{Cards}
\settowidth{\versewidth}{now I am too old to do otherwise}
\begin{verse}[\versewidth]
  A gun a chair a table some cards \\
  when I was young I had other plans \\
  A bullet myself a beer some cards \\
  now I am too old to do otherwise

  A sound a dead-man a glass a dead-man \\
  A sound a dead-man some cards \\
  you gonna use what's in your hands

  Silence dark dust some cards \\
  we are here only by chance \\
  Silence dark dust some cards \\
  we are here only by chance

  A rustle a light dust some cards \\
  life starts to spin around
\end{verse}

\newpage

\poemtitle{Minimal question to a decadent}
\settowidth{\versewidth}{a satellite guides guides}
\begin{verse}[\versewidth]
  how can a screen feed a child that dies \\
  can a friend in your page hear your cries \\
  a satellite guides guides your way \\
  satellite guides guides your way

  through life

  why do I need one hundred clicks to smile \\
  and how better will my life be \\
  after all I 've never found \\
  although I 've been searching around \\
  after all I 've never found \\
  in so many goods \\
  happiness  \\
  no happiness
\end{verse}

\end{document}
