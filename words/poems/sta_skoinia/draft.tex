% !TEX encoding = UTF-8
\documentclass[12pt]{article}
\usepackage[LGR]{fontenc}
\usepackage[greek,english]{babel}
\usepackage[utf8x]{inputenc}
\usepackage{verse}
\title{Στα σχοινιά}
\date{Αθήνα 2020}
\author{Γιώργος \latintext Timetrap}

\begin{document}
\maketitle

\poemtitle{Στα σκοινιά}
\begin{verse}
  Όλοι ψάχνουν μια θέση στον ήλιο

  μια αγκαλιά στου άγχους τα βράδια

  ένα καλό λόγο μετά την δουλειά

  ένα καλό λόγο για δουλειά

  Παράτα τα όλα και

  έλα ν' αγαπηθουμε

  στους δρόμους να γυρνάμε

  σαν αλητες στην βροχή

  Ζω σε κλουβί

  Όλη μου η ζωή

  Μαγκωμένος στα σκοινιά

  Προσπαθώ να κρατηθώ
\end{verse}


\poemtitle{Χωρίς τίτλο}
\begin{verse}
  πες αν θυμάσαι

  τo χρώμα

  της βροχής

  στην χώμα

  πες αν θυμάσαι

  την πρώτη πληγή

  στο σώμα

  και πες μου πως

  μπορείς να

  μένεις σύξυλος

  στην καταιγίδα

  πως μου αν θυμάσαι

  τους χτύπους της καρδιάς στον δρόμο
\end{verse}

\poemtitle{Ορχομενός}
\begin{verse}
  Συμπυκνωμένες πρακτικότητες

  Ψάχνουν μια στιγμή εκτόνωσης

  Έχουμε ζησει με συλλογικούς όρους

  Φευγαλέα

  Δεν το ξαναβρισκω πια

  Προσπαθήσαμε να σπάσουμε τα έπιπλα του ικεα

  Δέκα χρόνια μετά μας κοιτάνε γελώντας

  Θα γεράσω όταν κάνω παρατήρηση για φασαρία
\end{verse}

\poemtitle{τεστ}
\begin{verse}
  έχω μια ανατροφή

  από τόσο δα παιδί

  πως τους άλλους να χτυπώ

  να κοροϊδεύω να μισώ



  και ελξη μεγάλη

  στης τάξης τον πιο σιωπηλό

  στου σχολείου τον χαζό

  μεγάλωσα

  και έγινα πολιτικός
\end{verse}

\poemtitle{Ο φλύαρος}
\begin{verse}
  ολο μιλας πολυ

  θες να φυγεις μα εδω μενεις

  και όλα λες πολλά

  για όσα ξέρεις

  και δεν ξέρεις

\end{verse}

\poemtitle{Σύγχρονη λογοκρισία}
\begin{verse}
  κοσμικοί και πλούσιοι με εκατομμύρια ακολούθους

  οι υπόλοιποι ομιλητές σε άδεια ακροατήρια

  είσαι ελεύθερος να μιλήσεις

  να πεις ό,τι θες

  τίποτα δεν θα χρησιμοποιηθεί εναντίον σου

  κανείς δεν θα σε προσέξει

  κανείς δεν θα σε ακούσει

  κανείς δεν θα διαβάσει

  ό,τι και αν πεις

  πες ό,τι θες στο κενό

  σύχρονη λογοκρισία

  των μπάσταρδων καιρών

  δεν είναι η απουσία αλλά ο θόρυβος
\end{verse}

\poemtitle{Λίγο πριν το \latintext live}
\begin{verse}
  \greektext
  λίγο πριν το \latintext live

  \greektext
  λίγο σφίξιμο λίγο άγχος

  ψάξιμο για τουαλέτα τελευταία στιγμή

  συγκέντρωση τελευταία στιγμή

  λίγο πριν το \latintext live
\end{verse}


\poemtitle{Σαράκι}
\begin{verse}
  Μέρα με την μέρα

  μας τρώει το σαράκι

  κι όλο μας σέρνει

  πιο βαθια

  κι όλο μας σέρνει πιο βαθιά

  σε μίζερη άνεση

  με βρώμικο αερα

  σε άρρωστη τρέλα

  σαρωτική

  σε άρρωστη τρέλα σαρωτική

  Και όλοι ζητάμε

  μια θέση στον ήλιο

  μια αγκαλιά

  στου άγχους τα βράδια

  στου άγχους τα βράδια μια αγκαλιά

  σαφώς αποζητάμε

  έναν καθάριο λογο

  μια αιτία

  για αύριο στην δουλειά

  έναν λογο για δουλειά

  Και όλοι κρατάμε μια ανάσα πικρή

  μια ανάσα ζωη μες το κλουβί

  μαγκωμένοι χαμογελάμε στα σκοινιά

  προσπαθούμε να στεριώσουμε ξανά
\end{verse}

\poemtitle{Το ψυγείο}
\begin{verse}
\end{verse}

\poemtitle{Ρομπότ}
\begin{verse}
  Κάποτε θα ρθουν ρομπότ

  στην δουλειά και στην ζωή σου

  και θα τρέχουν τα πράματα

  με χρονοδιαγράμματα

  Κάποτε ήρθαν ρομποτ

  και το μόνο που φοβούνται

  λίγο πριν κοιμηθουν

  όνειρα καθόλου να μην δουν

  μηπως βραχυκυκλώσουν

  και μηπως μαλακώσουν

  να μην ονειρευτούν

  πως υπήρξανε παιδιά

  και αρχίσουνε τα κλάματα

  και μην βραχυκυκλώσουν

  να μην ονειρευτούν

  πως υπήρξαν παιδιά
\end{verse}

\poemtitle{Ρομπότ 2}
\begin{verse}
  Χθες το βράδυ είδα ένα ρομπότ

  παραπατούσε σε ευθείες γραμμές νεκρές

  στα μάτια είχε φλόγες εντολές

  σιγανά συνεχώς μονολογούσε

  (ρομπότ)

  το μαύρο το σκοτάδι αγαπώ

  όνειρα δεν θέλω πια να δω

  η ανάμνηση απρόσκλητη

  σαν έρχεται στον ύπνο

  με φέρνει αντιμέτωπο

  με ένα μικρό παιδί

  και ειναι καιρός που εξόριστος

  ειμαι από αυτήν την γη

  και ειναι καιρός που πέτρωσα

  και έχασα το κλειδί


  Χθές το βράδυ είδα ένα ρομπότ

  που ζωγράφιζε κουτάκια με ανθρωπους μέσα

  στα χέρια είχε πέτρες χαρακιές

  σιγανά συνεχώς μονολογούσε
\end{verse}

\poemtitle{Μηχανές}
\begin{verse}
 Κάποιοι ζουν σαν μηχανές

 Περπατούν σε γραμμές νεκρές

 Στα μάτια φλόγες εντολές

 Στα χέρια πέτρες χαρακιές

 Το μαύρο το σκοτάδι αγαπούν

  Όνειρα δεν θέλουν πια να δουν

  H ανάμνηση απρόσκλητη

  σαν έρχεται στον ύπνο

 τους φέρνει αντιμέτωπους

  με ένα μικρό παιδί

  Και είναι καιρός που εξόριστοι

  είναι από αυτήν την γη

  και είναι καιρός που πέτρωσαν

  και έχασαν το κλειδί
\end{verse}

\poemtitle{Πλάνο}
\begin{verse}


\end{verse}

\poemtitle{Λογική}
\begin{verse}
\end{verse}


\poemtitle{Οικογένεια}
\begin{verse}
  Πάμε μαζι σαν οικογένεια

  στον δρόμο, στην δουλειά και στην ζωή

  Και όταν πλουτίσω, θα είναι αγένεια

  στα μάτια να με κοιτάς

  Και όταν γεράσω θα ειναι ευγένεια

  στο χέρι να με φιλάς



  Και θα φροντίσω με την πρωτη

  αφορμή να σε πουλήσω

  Και θα φροντίσω με την πρώτη

  αφορμή να σε πουλήσω



  Στους δασκάλους που κοιτάνε

  και Στους μπάτσους που ρωτάνε

  Στ' αφεντικά που μουρμουράνε διαρκώς



  Μα τι έχει και δεν λέει

  μάλιστα το παιδί

  βρε τι έχει και δεν λέει

  διατάξτε δηλαδή


\end{verse}

\poemtitle{Οπλισμένη ανάγκη}
\begin{verse}
  Για όλα τα meetings

  που βαρέθηκες

  Για όλους γλείφτες

  που ανέχθηκες

  Για τις αργίες

  που δεν έζησες

  Τις απεργίες

  που εργάστηκες

  Τις απολύσεις

  που δεν μίλησες

  Και τις λύσεις

  που διαφώνησες

  Για τις βάρδιες

  που δεν πληρώθηκες

  Τις υπερώριες

  που σκοτώθηκες

  Είναι μια πράξη απελπισίας

  Εκμετάλλευσης και βίας

  Εκφοβισμού και αδυναμίας

  Κυνισμού και αφθονίας

  Οπλισμένων αναγκών

\end{verse}

\end{document}
